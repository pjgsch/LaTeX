%========================================
\input{texbas.inp}
%**********************************************************************
% Demo for LaTeX
%**********************************************************************
%
% This is the Preamble.
% It always starts with '\documentclass', which defines the 'style'.
% Here we use 'article', with the (default) option '10pt',
% defining the character size.
% There are more styles and options, which can be found in \LaTeX\ books.
%
% Following '\documentclass', packages can be loaded, which allow
% the use of special commands.
% Such packages have the extension .sty and must be available in
% the \LaTeX\ directory on your computer system.
% Only a few important packages and their options are shown in this
% example.

\newcommand{\DEM}{\MYTEXDIR/tpl}

\documentclass[10pt]{article}
\usepackage[english]{babel}

%======================================================================
% An empty line would cause LaTeX to make an indentation.
% The width of this indentation can be specified.
% Here we say that there must be no indentation at all.
% This allows us to use empty lines in our input files,
% which will only generate a new line feed.
%
% The pagestyle specifies which headings and footings are used.
% With the option 'empty' there will be no headings and footings.

\setlength{\parindent}{0em} 
\pagestyle{empty}

%**********************************************************************
% The document is started here.

\begin{document}
%======================================================================
% 
% The LaTeX source can be typed here.
% Although this is general practice, we follow another strategy.
% File with macros, styles and LaTeX source text are made separately
% included here with the command 'input'.
%
% Such a subdevision can be followed even deeper: input files may
% again contain other input files.
% This strategy leads to a clearer structure of documents and to a
% faster updating.
% The processing of the source leads to only one package of 'tex.*'
% files, instead of a bunch of files for each source text.
% Only relevant files (.ps mostly) may have to be saved under another
% name.
% However, generally they are printed on the spot and can be
% overwritten later on.
%
%======================================================================
% First the tile page is included.
% Although other pages may have page numbers and other header and
% footer text, this title page must have not.
% Also the page must not be seen as a page for the pagenumbering.

\input \DEM/demtit.txi
\thispagestyle{empty}
\addtocounter{page}{-1}

%======================================================================
% The table of contents comes after the title page.

\newpage
\tableofcontents

%======================================================================
% Several LaTeX source files are now included.
% We do not need extra packages.
% They constitute separate sections in our document, so the
% '\section' command is included first.
% Not wanted input can be commented out very easily.

\input \DEM/demlat.txi

\input \DEM/demmat.txi

\input \DEM/dembib.txi

\newpage
\appendix
\section{First appendix}  
\input \DEM/demfnt.txi
\newpage
\section{Second appendix} Empty.

%======================================================================
% The document is ended now.

\end{document}
%**********************************************************************
